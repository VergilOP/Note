\documentclass[a4paper]{article}
\usepackage[margin = 1 in]{geometry}
\usepackage{fancyhdr}
\usepackage{lastpage}
\usepackage{ctex}
\usepackage[utf8]{inputenc} % Required for inputting international characters
\usepackage[T1]{fontenc} % Output font encoding for international characters
\usepackage[sfdefault]{ClearSans} % Use the Clear Sans font (sans serif)
\usepackage{tocloft} 
\usepackage[hidelinks]{hyperref}
\usepackage{bmpsize}
\usepackage{graphicx}
\usepackage{epstopdf}
\usepackage{caption}
\usepackage{enumitem}
\usepackage{float}
\usepackage{multirow}
\usepackage{makecell}
\usepackage{amsmath} 

\author{Vergil/Zijun Li李子骏}
\title{Homework5 For Machine Learning}
\date{\vspace{-5ex}}

\begin{document}

\maketitle

\bf\begin{enumerate}

    \item In the game of Nim, consider a single pile with 14 chips. What is the Nimber of this position? Is
    this a P-position or an N-position?

    {\normalfont The Nimber of a single pile with 14 chips is 14. as Nimber of a position with a single pile of
    k chips, N(Pk) = k\\
    And it is a N-position(normal play)}

    \item Consider the sum of two games. Game 1 and game 2 are in positions with Nimber 4 and 6
    respectively. What is the Nimber of the combined game? Is this a P-position or an N-position?

    {\normalfont Nim = Nim(Game1) $\oplus$ Nim(Game2) = 0100 $\oplus$ 0110 = 0010\\
    It is a N-position}

    \item   Consider the MPC protocol for computing averages which we saw in the class. Suppose that
    the first and third student are colluding. Show that they can compute the age of the second
    student

    {\normalfont 
    What they have:\\
    A: A, X+R\\
    B: B, X+A+R\\
    C: C, X+A+B+R\\
    If A and B are colluding, B = (X+A+B+R) - (X+A+R). They could compute the age of the second student}

    \item  Consider the ZK proof for graph 3-coloring. What if the prover doesn't permute the colors in
    every iteration? That is, the prover permutes the colors once in the beginning and then sticks to
    that permutation for all iterations. Will the protocol still be zero-knowledge?

    {\normalfont No, it is not zero-knowledge, the verifier may have chance to gather enough information to infer the prover's specific coloring scheme}

\end{enumerate}

\end{document}