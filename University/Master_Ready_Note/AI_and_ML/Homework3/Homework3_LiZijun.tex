\documentclass[a4paper]{article}
\usepackage[margin = 1 in]{geometry}
\usepackage{fancyhdr}
\usepackage{lastpage}
\usepackage{ctex}
\usepackage[utf8]{inputenc} % Required for inputting international characters
\usepackage[T1]{fontenc} % Output font encoding for international characters
\usepackage[sfdefault]{ClearSans} % Use the Clear Sans font (sans serif)
\usepackage{tocloft} 
\usepackage[hidelinks]{hyperref}
\usepackage{bmpsize}
\usepackage{graphicx}
\usepackage{epstopdf}
\usepackage{caption}
\usepackage{enumitem}
\usepackage{float}
\usepackage{multirow}
\usepackage{makecell}
\usepackage{amsmath} 

\author{Vergil/Zijun Li李子骏}
\title{Homework3 For Machine Learning}
\date{\vspace{-5ex}}

\begin{document}

\maketitle

\bf\begin{enumerate}

    \item Consider the soda can game introduced in the class. Now suppose instead of a single circular table, there are two tables of identical size. A move consists of placing the can on any table (assuming there is a spot left on that table). Once placed, a can cannot move. The first one who cannot put a can on either table loses. In this game would you rather go first or second? Describe the winning strategy

    {\normalfont I would like to be the second.\\
    I can use mirroring strategy, place the can in a symmetrical position on another table }

    \item  In the take-chips-away game (with normal play rules):

    \subitem if the number of chips on the table is 15, does this represent a P-position or N-position? Describe the winning strategy.
    
    {\normalfont N-position\\
    Remove 3 and give 12 to the other player\\
    Then next player will give me either 11, 10, 9 and I give 8\\
    Then 4 and eventually 0, the next player wins}

    \subitem if the number of chips on the table is 8, does this represent a P-position or N-position? Describe the winning strategy

    {\normalfont P-position\\
    Give either 7, 6, 5 to the next player\\
    then previous player gives 4 and eventually 0, the previous player wins }

    \item  In the take-chips-away game (with Misère play rules):

    \subitem if the number of chips on the table is 15, does this represent a P-position or N-position? Describe the winning strategy.

    {\normalfont N-position\\
    Remove 2 and give 13 to the other player\\
    Then next player will give me either 12, 11, 10 and I give 9\\
    Then 5 and eventually 1, the next player wins}

    \subitem if the number of chips on the table is 8, does this represent a P-position or N-position? Describe the winning strategy

    {\normalfont N-position\\
    Remove 3 and give 5 to the other player\\
    Then next player will give me either 4, 3, 2 and I give 1, the next player wins}

    \item  In the modified take-chips-away game (with normal play rules), we only allow for a player to take out either 1 or 2 chips (rather than up to 3). For this game:

    \subitem if the number of chips on the table is 15, does this represent a P-position or N-position? Describe the winning strategy.
    
    {\normalfont P-position\\
    Give either 14, 13 to the next player\\
    then previous player gives 12, 9, 6, 3 and eventually 0, the previous player wins}

    \subitem if the number of chips on the table is 8, does this represent a P-position or N-position? Describe the winning strategy

    {\normalfont N-position\\
    Remove 2 and give 6 to the other player\\
    Then next player will give me either 5, 4 and I give 3\\
    Then eventually 0, the next player wins}

    \item Consider the game of Nim (with normal play rules):

    \subitem Consider piles position (3, 7) (i.e., two piles with the first pile having 3 and the second having 7 chips). Does this represent a P-position or N-position? Describe the winning strategy.

    {\normalfont N-position\\
    3 xor 7 \\= 011 xor 111 \\= 100\\
    the next player takes 4 from 7 (3,3), keep the xor result be 000 and eventually 000 xor 000 = 000, the next player wins}

    \subitem Consider piles position (5, 0, 3, 2). Does this represent a P-position or N-position and why? You don't need to describe the winning strategy

    {\normalfont N-position\\
    5 xor 0 xor 3 xor 2 \\= 101 xor 000 xor 011 xor 010 \\= 101 xor 011 xor 010 \\= 110 xor 010 \\= 100\\
    the next player takes 4 from 5 (1,0,3,2), keep the xor result be 000 and eventually 000 xor 000 xor 000 xor 000 = 000, the next player wins}

\end{enumerate}

\end{document}